\documentclass[11pt]{article}
\usepackage{amsmath,amssymb,amsthm}
\usepackage{algorithm}
\usepackage[noend]{algpseudocode} 
\usepackage{tabto}
\usepackage{mathtools}

%---enable russian----

\usepackage[utf8]{inputenc}
\usepackage[russian]{babel}

% PROBABILITY SYMBOLS
\newcommand*\PROB\Pr 
\DeclareMathOperator*{\EXPECT}{\mathbb{E}}


% Sets, Rngs, ets 
\newcommand{\N}{{{\mathbb N}}}
\newcommand{\Z}{{{\mathbb Z}}}
\newcommand{\R}{{{\mathbb R}}}
\newcommand{\Zp}{\ints_p} % Integers modulo p
\newcommand{\Zq}{\ints_q} % Integers modulo q
\newcommand{\Zn}{\ints_N} % Integers modulo N

% Landau 
\newcommand{\bigO}{\mathcal{O}}
\newcommand*{\OLandau}{\bigO}
\newcommand*{\WLandau}{\Omega}
\newcommand*{\xOLandau}{\widetilde{\OLandau}}
\newcommand*{\xWLandau}{\widetilde{\WLandau}}
\newcommand*{\TLandau}{\Theta}
\newcommand*{\xTLandau}{\widetilde{\TLandau}}
\newcommand{\smallo}{o} %technically, an omicron
\newcommand{\softO}{\widetilde{\bigO}}
\newcommand{\wLandau}{\omega}
\newcommand{\negl}{\mathrm{negl}} 

% Misc
\newcommand{\eps}{\varepsilon}
\newcommand{\inprod}[1]{\left\langle #1 \right\rangle}

 
\newcommand{\handout}[5]{
  \noindent
  \begin{center}
  \framebox{
    \vbox{
      \hbox to 5.78in { {\bf Научно-исследовательская практика} \hfill #2 }
      \vspace{4mm}
      \hbox to 5.78in { {\Large \hfill #5  \hfill} }
      \vspace{2mm}
      \hbox to 5.78in { {\em #3 \hfill #4} }
    }
  }
  \end{center}
  \vspace*{4mm}
}

\newcommand{\lecture}[4]{\handout{#1}{#2}{#3}{ФИО: #4}{Некоторые свойства Фи-функции #1}}

\newtheorem{theorem}{Теорема}
\newtheorem{lemma}{Лемма}
\newtheorem{definition}{Определение}
\newtheorem{corollary}{Следствие}
\newtheorem{fact}{Факт}

% 1-inch margins
\topmargin 0pt
\advance \topmargin by -\headheight
\advance \topmargin by -\headsep
\textheight 8.9in
\oddsidemargin 0pt
\evensidemargin \oddsidemargin
\marginparwidth 0.5in
\textwidth 6.5in

\parindent 0in
\parskip 1.5ex


% main
\begin{document}

\lecture{}{Лето 2020}{}{Щепецков Леонид}
При данных ограничениях мы находим, что желаемая сумма равна:
\[1 + 7 + 11 + 13 + 17 + 19 + 23 + 29 = 120 = \frac{1}{2} \cdot 30 \cdot 8. \]
Это хороший момент, чтобы дать применение формулы инверсии Мёбиуса.

\begin {theorem} 
\label{th:7-8}
\emph{Для любого положительного числа n,}
\[ \phi (n) = n \sum_{d|n} \mu(d)/d \]
\end {theorem}
\begin {proof} 
Доказательство обманчиво просто: если применить формулу инверсии к
\[F(n) = n = \sum_{d|n} \phi(d),\]
результатом будет
\[\phi(n) = \sum_{d|n} \mu(d)F(n/d) = \sum_{d|n}\mu(d)n/d.\]
\end {proof}

Давайте рассмотрим ситуацию, когда $n = 10$ снова. Легко можно заметить, что
\begin{align*}
10 \sum_{\alpha|10}\mu(d)/d &= 10 [\mu(1) + \mu(2)/2 + \mu(5)/5 + \mu(10)/10] \\ 
&=  10[1 + (-1)/2 + (-1)/5 + (-1)^2/10] \\ 
&=  10[1 - 1/2 - 1/5 + 1/10] = 10 \cdot 2/5 = 4 = \phi(10).
\end{align*}
Начиная с Теоремы~\ref{th:7-8}, легко определить значение Фи-Функции для любого положительного целого числа $n$. Предположим, что разложение простой степени $n$ это $n = p_1^{k_1}p_2^{k_2} \dots p_r^{k_r}$ и рассмотрим продукт
\[ P = \prod_{p_i|n} \big(\mu(1)+\mu(p_i)/p_i + \dots + \mu(p_i^{k_i})/p_i^{k_i}\big). \]
Умножив это, получим сумму членов вида
\[\mu(1)\mu(p_1^{a_1})\mu(p_2^{a_2}) \dots \mu(p_r^{a_r})/p_1^{a_1}p_2^{a_2} \dots p_r^{a_r}, \quad  0 \leq a_i \leq k_i\] 

или, так как известно, что $\mu$ мультипликативно,
\[\mu(p_1^{a_1}p_2^{a_2}\dots p_r^{a_r}/p_1^{a_1}p_2^{a_2} \dots p_r^{a_r} = \mu(d)/d,\]
где суммирование происходит по множеству делителей $d = p_1^{a_1}p_2^{a_2}\dots p_r^{a_r}$ по $n$. Следовательно, $P = \sum_{d|n} \mu(d)/d$. Это следует из Теоремы~\ref{th:7-8}, что
\[ \phi(n) = n \sum_{d|n} \mu(d)/d = n \prod_{p_i|n} \big(\mu(1)+\mu(p_i)/p_i + \dots + \mu(p_i^{k_i})/p_i^{k_i}\big). \]
Но $\mu(p_i^{a_i}) = 0 $ всякий раз, когда $a_i\geq 2.$ Как результат, последнее записанное уравнение сводится к
\[\phi(n) = n \prod_{p_i|n}(\mu(1)+\mu(p_i)/p_i) = n \prod_{p_i|n}(1-1/p_i),\]
что согласуется с формулой, установленной ранее другими рассуждениями. Что существенно в этом аргументе, так это то, что не делается никакого предположения относительно мультипликативного характера фи-функции, только о $\mu$

\begin{center}
    \textbf{\Large ЗАДАЧИ 7.4}
\end{center}


\begin{enumerate} 
  \item \label{pr:1}Для натурального числа $n$, докажите что
    \begin{align*}
        \sum_{d|n}(-1)^{n/d}\phi(d) =  
            \begin{cases}
                0 &\text{если $n$ чётный} \\
                $-n$ &\text{если $n$ нечётный} 
            \end{cases}
    \end{align*}
    [Подсказка: Если $n = 2^kN$, где $N$ нечётный, тогда $\sum_{d|n}(-1)^{n/d}\phi(d) = \sum_{d|2^{k-1}N}\phi(d) - \sum_{d|N}\phi(2^kd).$]
  \item Подтвердите, что $\sum_{d|36}\phi(d) = 36$ и $\sum_{d|36}(-1)^{36/d}\phi(d) = 0.$
  \item Для положительного числа $n$, докажите, что $\sum_{d|n}\mu^2(d)/\phi(d) = n/ \phi(n).$ [Подсказка: Смотри подсказку в Задаче~\ref{pr:1}.]
  \item Используя Задачу 3 из Главы 6.2, дайте общее доказательство факта, что
  \[n \sum_{d|n} \mu(d)/d = \phi(n).\]
  \item Пусть факторизация числа $n > 1$ равна $n = p_1^{k_1}p_2^{k_2} \dots p_r^{k_r}$, тогда установите:
    \begin{enumerate}
        \item \label{pr:a}\[ \sum_{d|n}\mu(d)\phi(d) = (2 - p_1)(2 - p_2)\dots (2 - p_r) \]
        \item \[ \sum_{d|n}d\phi(d) = \left(\frac{p_1^{2k_1+1} + 1}{p_1+1}\right)\left(\frac{p_2^{2k_2+1} + 1}{p_2+1}\right)\dots \left(\frac{p_r^{2k_r+1} + 1}{p_r+1}\right) \]
        \item \[ \sum_{d|n}\phi(d)/d = \left(1 + \frac{k_1(p_1 - 1)}{p_1}\right)\left(1 + \frac{k_2(p_2 - 1)}{p_2}\right) \dots \left(1 + \frac{k_r(p_r - 1)}{p_r}\right) \]
        [Подсказка: Для части~\ref{pr:a}, используйте Задачу 3 из Главы 6.2.]
    \end{enumerate}
  \item Верифицируйте формулу $ \sum_{d = 1}^n \phi(d)[n/d] = n(n + 1)/2 $ для любых положительных чисел $n$.
  [Подсказка: Это прямое применение теорем 6-11 и 7-6.]
  \item Если $n$ это свободное от квадрата число, докажите что $\sum_{d|n} \sigma(d^{k-1})\phi(d) = n^k$ для всех чисел $k \geq 2$.
  \item  Для свободных от квадратов чисел $n > 1$, покажите что $\tau(n^2) = n$ тогда и только тогда, когда $n = 3$.
  \item Докажите, что $3 | \sigma(3n+2)$ и $4 | \sigma(4n + 3)$ для любых положительных чисел $n$. 
  \item
    \begin{enumerate}
        \item Дано $k > 0$, установите, что существует последовательность из $k$ последовательных чисел $n+1, n+2, \dots , n+k$, удовлетворяющих
        \[\mu (n+1) = \mu (n+2) = \dots = \mu (n+k) = 0.\]
        [Подсказка: Рассмотрим систему линейных уравнений
        \[ x \equiv -1 (mod 4), x \equiv -2 (mod 9), \dots , x \equiv -k (mod p_k^2) \]
        где $p_k$ это k-тое простое число]
        \item Найти четыре последовательных числа для $\mu(n) = 0.$
    \end{enumerate}
  \item Доказать утверждения ниже:
    \begin{enumerate}
        \item Число $n$ простое тогда и только тогда, когда $\sigma(n) + \phi(n) = n_{\tau}(n)$. [Подсказска: Во-первых выведите соотношение $\sum_{d|n} \sigma(d)\phi(n/d) = n_{\tau}(n).$]
        \item Число $n$ простое тогда и только тогда, когда $\phi(n) | n - 1$ и $ n + 1 | \sigma(n)$. [Подсказка: Посмотрите Задачу 11(а) из Главы 7-2.]
    \end{enumerate}
  \item Докажите, что существует бесконечно много чисел $n$, которые удовлетворяют $\phi(n) = n/3$, но ни одного, удовлетворяющего $\phi(n) = n/4$
  \item Для $n > 2$, установить соотношение $\phi(n^2) + \phi((n+1)^2) \leq 2n^2$.
\end{enumerate}

\end{document}
